\subsection{Purpose of Usability Evaluation}
    The purpose of the usability evaluation is to assess how effectively the company (Track Genesis) can use our CV to upload, rank and evaluate received CVs thereby making the requirement process easier and faster. The goal is to make the CV analyser user friendly, efficient, and meeting all the needs of the company.
\subsection{Definition of Usability}
    Usability can be defined as "The extent to which a product can be used by specified users to achieve specified goals with effectiveness, efficiency, and satisfaction in a specified context of use" \parencite{iso9241}(ISO 9241-11, 1998). To attain optimal usability, our CV is intuitive, easy to learn and provides satisfactory experience for the user.
\subsection{Usability Factors}
\begin{itemize}
    \item \textbf{Intuitive design} – Navigation and key features such as uploading and ranking should be clear, thereby making it effortless for the user to upload and rank CVs
    \item \textbf{Ease of learning} – New customers should not need a lot of training before being able to use the program. Clear instructions should be given for key features.
    \item \textbf{Efficiency of use} – The CV analyser should process CVs efficiently using AI driven mechanism to highlight the most important candidates.
    \item \textbf{Error Prevention and recovery} – The program should minimize the possibility of an error such as incorrect file upload and if an error does happen, clear feedback should be provided along with how to resolve the error.
    \item \textbf{Subjective satisfaction} – The overall experience should be smooth and hitch free, making the user feel confident in the result provided.
\end{itemize}
\subsection{Application Recruitment}
    Our analyser evaluates CVs based on predefined criteria such as experience, skills and job relevance, providing hiring managers with the appropriate candidates. The program aims to streamline the hiring process, save time, and improve hiring efficiency.