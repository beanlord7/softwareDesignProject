\chapter{Project Requirements}    
    Our client has requested us to design a software solution that simplifies and speeds up their recruitment process by automating CV review and sorting and providing a ranked list of CV summaries. Based on the Live Brief provided by TrackGenesis and the following presentation, we have decided on what we believe to be the essential requirements for this software. 

    \section{Functional Requirements}
        The following list is the minimum and essential requirements needed for the system to function correctly:
        \begin{enumerate}
            \item \textbf{Job profile:} The program must allow the user to set up a job profile to use as a working environment for adding and keeping a job description and candidate resumes. The reason why we decided to implement profiles was to enable an easy way to save and resume work between multiple sessions, including adding resumes from new applicants to a previously generated ranking list without having to re-upload all of them. We have also made a reasonable assumption that the company might be hiring for multiple positions at the same time, and the user could benefit from having multiple saved job descriptions that they could later reuse without having to manage their storage themselves.
            \item \textbf{Job description data input:} The user should be able to add a job description to a job profile either through a text field as plain text, or by uploading a PDF, DOCX, or TXT file.
            \item \textbf{Job description parsing:} Because the client indicated optimisation and automation as their priorities, we have decided to use NLP to parse job descriptions to extract desired criteria for each job, instead of having the user set up their own criteria manually. This should simplify and hasten the setup of new job profiles. Parsing of job descriptions should result in a set of criteria which are then stored for each profile.
            \item \textbf{Resume data input:} The user needs to be able to upload multiple CV files in different formats (PDF, DOCX, TXT) at the same time, which are then saved on the profile and assigned an candidate ID. The resumes should be stored in case the user wants to view them, and to avoid the need of uploading them multiple times in case they need to be parsed again (e.g. if a job description is updated).
            \item \textbf{Resume data parsing:} CV files for each candidate need to have text extracted before it can be parsed into data that's usable by our data matching and ranking system. The workflow of handling resumes and job descriptions needs to include these steps to cover the file formats requested by our client:
                \begin{enumerate}
                    \item \textbf{Plain text extraction from CV files:} this can be achieved by reading TXT files natively in Java, using the \textit{Apache PDFBox} library for PDF files, and the \textit{Apache POI} library for DOCX files. However, the \textit{Apache Tika} library acts as a wrapper for aforementioned libraries and can detect all 3 file types and extract text from all of them using just one function.
                    \item \textbf{Natural Language Processing:} Extracted text is then passed on for processing via the \textit{Stanford CoreNLP} library. Ultimately, processed data has to result in \textit{entities} (e.g. Name, Programming Languages, Company) with assigned \textit{attributes} (e.g. John, Python, Google) that are then put into \textit{categories} (e.g. Personal Info, Skills, Work Experience).
                \end{enumerate}
            \item \textbf{Data matching and ranking:} Parsed data needs to be stored in individual JSON files and then compared to the job description JSON file that has the same structure to hold entities and attributes. The CV files are assigned a \textit{relevance score}, the value for each match is determined by it being an exact or partial match.
            \item \textbf{Result summary display and storage:} The program needs to be able to save data of each job profile and resume into JSON files that are encrypted using \textit{AES-256 encryption}, and then accessed by the program and displayed in the GUI as a list that is ranked by the relevance score.
        \end{enumerate}
    \section{Non-functional Requirements}
        These are additional requirements that do not affect the basic functionality, but alter it in ways that provide additional benefits to the User.
        \begin{enumerate}
            \item \textbf{Security:} Because TrackGenesis is a Scotland based company, and due to the nature of data being gathered and processed by the program, we have made the assumption that it should be compliant with the \textcite{ukgdpr2025} and encrypt parsed data before it is stored. Therefore we decided to implement \textit{AES-256 encryption} using \textit{Java Cryptography Architecture (JCA)} as it provides an easily accessible built-in way to encrypt the JSON data \textit{(Oracle, 2025)}.
            \item \textbf{Usability:} The program should have an intuitive GUI that allows the user to easily set up job descriptions, upload CV files and view ranked summaries. We decided to implement a GUI is to avoid users having to familiarise and remember terminal commands, which is less intuitive and slower than a very basic GUI. This should improve the speed of resume screening further, which is one of the main stated goals of the client.
            \item \textbf{Scalability and performance:} The user should be able to upload a large amount of complex CV files at the same time and not experience a major slowdown in data processing. Already processed data should be saved on the job profile and not processed repeatedly if additional CVs need to be uploaded. The data should be well organised and presented in a way that's intuitive and easy to navigate.
        \end{enumerate}
    
