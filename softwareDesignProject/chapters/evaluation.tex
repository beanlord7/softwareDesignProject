\chapter{Evaluation}
One part of designing the project, which presented a significant challenge, was selecting the correct tools for our purposes, primarily the Java libraries required for extracting raw text from resumes and then parsing the data via NLP.

At first, we were looking into the recommendations presented to us via the Live Brief, which included some Python libraries. After doing some research on those, we were informed that despite the Live Brief, we were required to use Java exclusively. This was a slight setback for us, but it simulated a very real possibility of requirements changing mid-project and shows the importance of careful requirements gathering.

Our Client emphasised efficiency and automation as a requirement, so we needed to pick Java libraries that were fast, accurate, and able to process a large number of complex resumes. Plain text extraction from common file formats like PDF and DOCX is quite straightforward and fast, so we went with Apache Tika as recommended in the Live Brief. However, the NLP solution required more consideration. After doing some research, we have decided to use Stanford CoreNLP as recommended in the Live Brief. OpenNLP would offer easy integration with Tika (both developed by the Apache Software Foundation), however, multiple sources regard CoreNLP as the faster and more accurate libraries according to their benchmarks \parencite{Schmitt2019,Nanavati2015}. Having gone through these considerations we are therefore confident in our choice of tools for the purposes of this program.

Another challenge that presented itself during the design stage was achieving compliance with the requirements whilst maintaining a high level of cohesion in-between so many different parts of the overall design being done by different people. It is clear that achieving this requires constant communication and a shared workspace like MS Teams or GitHub, which our team opted out of. 

In our evaluation we are able to see some inconsistencies, which we deem to be important, but in totality do not detract from the overall design. In the end we believe that we succeeded in designing a system that achieved the most important requirements. Our final design turned out to be minimalistic and efficient at fulfilling its main functions. It requires very little work from the user due to how we implemented automation and streamlined the UI. Thus, we are content with the final design and confident that it will satisfy the needs of TrackGenesis.